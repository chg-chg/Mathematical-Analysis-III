%!TEX TS-program = xelatex
\documentclass[xetex]{beamer}

\usefonttheme{professionalfonts}

\usepackage[UTF8]{ctex}
\usepackage{hyperref}
\usepackage{unicode-math}
\usepackage{amsmath, amssymb}
\usepackage{graphicx, wrapfig}

\usepackage{nopageno}

\graphicspath{{./img/}}

\usetheme[block=fill, subsectionpage=progressbar]{metropolis}

\setmathfont{XITS Math}

\begin{document}
	\begin{frame}
		\title{各种积分间的联系与场论初步}
		\subtitle{各种积分间的联系}
		\author{数学分析MOOC小组}
		\date{  }
		\titlepage
	\end{frame}
	
	\begin{frame}
		\frametitle{概览}
		\begin{itemize}
			\item[1.] 知识回顾		
			\item[2.] 习题讲解
		\end{itemize}
	\end{frame}
	
	\begin{frame}
		\section{知识回顾}
	\end{frame}
	
	\begin{frame}
		\frametitle{积分与路径无关的条件}
		\textbf{定理22.4}
			设$D$是平面单连通区域,函数$P(x, y)$,$Q(x, y)$在$D$上有连续偏导数,则下列四则断言等价:
			\begin{itemize}
				\item[(1)] 设$D$中任一逐段光滑的闭曲线$L$,有
					\begin{equation*}
						\oint_L P\mathrm{d}x + Q\mathrm{d}y = 0
					\end{equation*}
				\item[(2)]对$D$中任一逐段光滑的闭曲线$L$,曲线积分
					\begin{equation*}
						\int_L P\mathrm{d}x + Q\mathrm{d}y
					\end{equation*}
					与路径无关,只与$L$的起点与终点有关;
				\item[(3)]微分式$P\mathrm{d}x + Q\mathrm{d}y$在$D$内是某函数$u(x, y)$的全微分,即有
					\begin{equation*}
						\mathrm{d}u = P\mathrm{d}x + Q\mathrm{d}y
					\end{equation*}
				\item[(4)] 在$D$内每一点有
					\begin{equation*}
						\frac{\partial P}{\partial y} = \frac{\partial Q}{\partial x}
					\end{equation*}
			\end{itemize}
	\end{frame}
	
	\begin{frame}
		\frametitle{奇点的存在}
		使得$\frac{\partial P}{\partial y} = \frac{\partial Q}{\partial x}$不成立的点称为积分$\int_L P\mathrm{d}x + Q\mathrm{d}y$的奇点。在复连通区域$D$中条件$\frac{\partial P}{\partial y} = \frac{\partial Q}{\partial x}$,这时有以下结论:
		\begin{itemize}
			\item[(1)]对$D$内任意一条不包围奇点的闭曲线$L$,有
				\begin{equation*}
					\oint_L P\mathrm{d}x + Q\mathrm{d}y = 0
				\end{equation*}
			\item[(2)] 环绕某一奇点的任意两条简单闭曲线$L_1$和$L_2$的正向的积分相等,即
				\begin{equation*}
					\oint_{L_1} P\mathrm{d}x + Q\mathrm{d}y = \oint_{L_2} P\mathrm{d}x + Q\mathrm{d}y
				\end{equation*}
				这个公共值称为该奇点的循环常数。
		\end{itemize}
	\end{frame}
	\begin{frame}
		\frametitle{奇点的存在}
		\begin{itemize}
			\item[(3)]环绕某一奇点$n$圈的光滑闭曲线$L$,其中$n_1$圈是正向,$n_2$圈是负向,$n_1 + n_2 = n$,则积分
			\begin{equation*}
			I = \oint_L P\mathrm{d}x + Q\mathrm{d}y
			\end{equation*}
			等于该奇点循环常数的$n_1 - n_2$倍。
			\item[(4)]若不自相交光滑闭曲线$L$包围了$k$个奇点,则沿$L$正向的积分
			\begin{equation*}
			I = \oint_L P\mathrm{d}x + Q\mathrm{d}y
			\end{equation*}
			等于这$k$个奇点循环常数的和。
		\end{itemize}
	\end{frame}
	\begin{frame}
		\frametitle{积分与路径无关的条件:三维的情形}
		\textbf{定理22.5}
			设$V$是空间的单连通区域,函数$P(x, y)$,$Q(x, y)$,$R(x, y)$在$V$上有连续偏导数,则下列四则断言等价:
			\begin{itemize}
				\item[(1)] 沿$V$中任一逐段光滑的闭曲线$L$,有
				\begin{equation*}
				\oint_L P\mathrm{d}x + Q\mathrm{d}y + R\mathrm{d}z = 0
				\end{equation*}
				\item[(2)]对$V$中任一逐段光滑的闭曲线$L$,曲线积分
				\begin{equation*}
				\int_L P\mathrm{d}x + Q\mathrm{d}y+ R\mathrm{d}z
				\end{equation*}
				与路径无关,只与$L$的起点与终点有关;
				\item[(3)]$P\mathrm{d}x + Q\mathrm{d}y+ R\mathrm{d}z$在$V$内是某函数$u(x, y, z)$的全微分,即有
				\begin{equation*}
				\mathrm{d}u = P\mathrm{d}x + Q\mathrm{d}y + R\mathrm{d}z
				\end{equation*}
				\item[(4)] 在$V$内每一点有
				\begin{equation*}
				\frac{\partial P}{\partial y} = \frac{\partial Q}{\partial x}, 
				\frac{\partial Q}{\partial z} = \frac{\partial R}{\partial y},
				\frac{\partial R}{\partial x} = \frac{\partial P}{\partial z}
				\end{equation*}
			\end{itemize}
			
	\end{frame}
	
	\begin{frame}
		\section{习题讲解}
	\end{frame}
	
	\begin{frame}
		\frametitle{习题1(1)}
		验证以下积分与路径无关,并求它们的值
		\begin{equation*}
			\int_{(0,0)}^{(1,1)} (x - y) (\mathrm{d}x - \mathrm{d}y)
		\end{equation*}
		
		\textbf{解:}
			\begin{equation*}
			  \begin{split}
			    &\int_{(0,0)}^{(1,1)} (x - y) (\mathrm{d}x - \mathrm{d}y) \\
			  = &\int_{(0,0)}^{(1,1)} (x - y)\mathrm{d}x + (y- x)\mathrm{d}y \\
			  = &\int_{(0, 0)}^{(1, 0)}  (x - y)\mathrm{d}x + \int_{(1, 0)}^{(1, 1)} (y - x) \mathrm{d}y\\
			  = &0.5 - 0.5\\
			  = &0\\
			  \end{split}
			\end{equation*}
	\end{frame}
	
	\begin{frame}
		\section{谢谢大家!}
	\end{frame}
\end{document}