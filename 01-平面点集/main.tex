%!TEX TS-program=xelatex
\documentclass[xetex]{beamer}
% 将上面这一行修改成下面这个样子,可以创建适合于发布的版本,这去除了所有的动画
% \documentclass[xetex, handout]{beamer}

% 规范注意:

% 使用正确的主题(beamer主题、文字字体)
% 使用正确的title信息(title、subtitle、author、date)
% 合理使用frame 和 standout frame
% 块(block、exampleblock、alertblock)以及文字段落内部的\alert的使用
% 使用图片需要使用\begin{figure}...\end{figure}并附带\caption和\label信息
% 使用enumerate和itemize组织你的点
% 使用section给你的幻灯片分部分
% 公式,文字段落内嵌公式和单独的公式块的使用
% \DeclareMathOperator的使用,以及学会在网上查找你不知道怎么输入的数学符号
% 动画的使用
% 讲课录制使用什么版本的文档;对外发布使用什么版本的文档(handout)

\usefonttheme{professionalfonts}

\usepackage[UTF8]{ctex}
\usepackage{hyperref}
\usepackage{unicode-math}
\usepackage{amsmath, amssymb}
\usepackage{graphicx, wrapfig}

\usepackage{nopageno}

\DeclareMathOperator{\argmax}{argmax}

\usetheme[block=fill]{metropolis}

\setmathfont{XITS Math}

\title{多元函数的极限与连续性}
\subtitle{平面点集}
\author{数学分析MOOC小组 }
\date{}

\begin{document}

\frame{\maketitle}

\begin{frame}
    \frametitle{概览}
	
    \begin{enumerate}
        \item 知识回顾
			\begin{enumerate}
           \item 平面点集和邻域的表示
			\item 点与点集的关系
			\item 几种重要的平面点集
			\item 其他知识
        	\end{enumerate}
        \item 习题讲解
        
    \end{enumerate}


\end{frame}

\section{知识回顾}
\begin{frame}
    \frametitle{平面点集和邻域的表示} 
  
平面上的点集$E$可以用集合表示成:\\$\qquad E=\{(x,y)\mid(x,y)\mbox{满足某性质}\}$\\ 
$P_{0}(x_{0},y_{0})$的 $\delta$ 圆邻域可以表示为:\\$\qquad O(P_{0},\delta)= \{ (x,y)\mid(x-x_{0})^2+(y-y_{0})^2<\delta ^2\}$
	
	
\end{frame}

\begin{frame}
    \frametitle{点与点集的关系} 
  
    \begin{enumerate}
\item 若$\exists \delta >0$,使得$O(P_{0},\delta) \subset E$,则称$P_{0}$为$E$的内点
\item 若$\exists \delta >0$,使得$O(P_{0},\delta) \cap E= \emptyset$,则称$P_{0}$为$E$的外点
\item 若$\forall \delta >0$,使得$O(P_{0},\delta) \cap E\ne \emptyset$,且$O(P_{0},\delta) \backslash E \ne \emptyset$则称$P_{0}$为$E$的边界点
\item 若$\forall \delta >0$,使得$O^* (P_{0},\delta) \cap E \ne \emptyset$,则称$P_{0}$为$E$的聚点
	
    \end{enumerate}
\end{frame}

\begin{frame}
    \frametitle{几种重要的平面点集} 
    \begin{enumerate}
 	\item[(1)]\textbf{开集}\qquad 称平面点集$E$是开集,如果$E$的所有点都是内点。
 	\item[(2)]\textbf{闭集}\qquad 称平面点集$E$是闭集,如果$E$的所有聚点(如果有的话)属于$E$。
 	\item[(3)]\textbf{连通集}\qquad 称$E$是连通集,如果$E$的任意两点都能用完全含于$E$的有限条直线段组成的折线连接起来。
 	\item[(4)]\textbf{区域}\qquad 连通的开集称为开区域,或简称为区域。
 	\item[(5)]\textbf{闭区域}\qquad 区域连同它的边界点所组成的集合称为闭区域。
    \end{enumerate}
  
\end{frame}

\begin{frame}
    \frametitle{其他知识} 
  \begin{enumerate}
\item 平面点列的极限较为简单
\item 柯西收敛定理等并非重点掌握内容,同学们酌情掌握
	
    \end{enumerate}
	
\end{frame}

\section{习题讲解}
\begin{frame}
    \frametitle{习题3.6} 
   3.判别下列平面点集哪些是开集、闭集、有界集或区域,并分别指出它们的聚点:\\
	(6)$E=\left\{(x,y)\mid y = \sin\frac{1}{x},x>0 \right\}$ 
	\\解:\\聚点:$\left\{(x,y)\mid y = \sin\frac{1}{x},x>0 \mbox{或}x=0,|y|\le 1\right\}$\\
所有$E$上的点都是边界点(都不是内点),所以$E$不是开集;并不是所有的$E$的聚点都属于$E$,所以$E$不是闭集,自然也不是区域;$x$没有范围限制,所以$E$不是有界集。
\end{frame}

\begin{frame}
    \frametitle{习题4} 
   4.设$F$为闭集,$G$为开集,证明$F \,\backslash\, G$是闭集,$G \,\backslash\, F$是开集。
	\\解:\\$R^2$为全集,记$F^c$和$G^c$分别为$F$和$G$的补集;$F \,\backslash\, G=F\cap G^c$,$G \, \backslash\, F=G\cap F^c$;
	我们先证$F^c$为开集,$G^c$为闭集:\\
	(1)$\, \forall P_{0} \in F^c$,则必有$P_{0}$是$F^c$的内点,即满足$\exists \delta >0$,使得$O(P_{0},\delta) \subset F^c$。(这是因为若$P_{0}$不是$F^c$的内点,则有$\forall \delta >0$,均使得$O(P_{0},\delta ) \not\subset F^c$,则$O(P_{0},\delta) \cap F \ne \emptyset$。又因为$P_{0}\not\in F$,所以$O^*(P_{0},\delta) \cap F \ne \emptyset$,即$P_{0}$是闭集$F$的聚点,则$P_{0}\in F$,矛盾。)所以$F^c$是开集。\\
	(2)对于$\, \forall G^c$的聚点$P_{0}$,有$\forall \delta >0$,$O(P_{0},\delta) \cap G^c \ne \emptyset$,则$P_{0}\not\in G$(这是因为若$P_{0}\in G$,则$\exists \delta >0$,使得$O(P_{0},\delta) \subset G$,此时$O(P_{0},\delta) \cap G^c = \emptyset$,矛盾),所以$P_{0}\in G^c$,所以$G^c$是闭集。
\\下证$F\cap G^c$是闭集,$G\cap F^c$是开集:
	
\end{frame}
\begin{frame}
    \frametitle{习题4} 
下证$F\cap G^c$是闭集,$G\cap F^c$是开集:\\
(3)记$E=F\cap G^c$。若$P_{0}$是$E$的聚点,则有$\forall \delta >0$,$O(P_{0},\delta) \cap E \ne \emptyset$,即一方面有$\forall \delta >0$,$O(P_{0},\delta) \cap F \ne \emptyset$,所以$P_{0}$是$F$的聚点,则$P_{0}\in$闭集$F$;同理,另一方面有$\forall \delta >0$,$O(P_{0},\delta) \cap G^c \ne \emptyset$,所以$P_{0}$是$G^c$的聚点,则$P_{0}\in$闭集$G^c$。所以$P_{0}\in F\cap G^c$,即$P_{0}\in E$,这便证明了$F\cap G^c$是闭集。\\
(4)若$P_{0}\in G\cap F^c$,则一方面$P_{0}\in$开集$G$,所以$\exists\delta_1>0$,使得$O(P_{0},\delta_1) \subset G$;另一方面$P_{0}\in$开集$F^c$,所以$\exists\delta_2>0$,使得$O(P_{0},\delta_2) \subset F^c$;取$\delta=\min(\delta_1,\delta_2)$,则有$O(P_{0},\delta) \subset G\cap F^c$,即$P_{0}$是$G\cap F^c$的内点。这便证明了$G\cap F^c$是开集。
\end{frame}
 

\begin{frame}[standout]
谢谢大家!
\end{frame}

 

\end{document}